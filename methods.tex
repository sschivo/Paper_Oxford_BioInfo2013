\subsection{Modelling abstractions}\label{subsec:abstractions}

In molecular networks in the cell, cascades of chemical and physical interactions enable propagation of 
signals through the network. In this process, the activity of upstream molecules induces a change in the 
concentration or activity of downstream molecules. For many reactions, the values of the kinetic parameters 
are unknown or difficult to collect. This lack of knowledge hampers the feasibility 
of computational models that describe molecular networks in fine mechanistic detail, especially for larger networks.
As a solution to this problem, we propose the construction of models at a higher level of abstraction, 
thereby reducing the number of parameters involved. In choosing a suitable abstraction level, it is important to 
retain enough descriptive power to be an asset in the study of biological networks.

As a first abstraction in ANIMO models, the active and inactive forms of each network component 
are represented together by a single node in the network.
Each of these nodes is characterized by its \emph{activity level}, which
is a representation of the fraction of active molecules of that species. Activity levels are 
discretized into integer variables with a user-defined granularity, ranging from Boolean (2 levels) to 
near-continuous (100 levels).

Detailed biochemical reaction mechanisms are abstracted to \emph{interactions}. These interactions preserve 
cause-and-effect relationships between molecules and can represent either activations ($\rightarrow$) 
or inhibitions ($\dashv$\,). Each occurrence of the interaction raises or lowers the activity level 
of the downstream node by one discrete step, respectively. This aggregation of elementary reactions 
into single interaction steps, reducesing the number of kinetic parameters involved, while preserving cause-and-effect relationships.
For example, consider a reaction in which enzyme $E$ activates substrate kinase $S$, transferring
a phosphate group from a molecule of ATP to a molecule of $S$. In biochemical
terms, the reaction can be represented as
$$
\mbox{\it E} + \mbox{\it S} + \mbox{\it ATP} \rightleftarrows \mbox{\it ES} + \mbox{\it ATP} \rightarrow \mbox{\it ES}^{\mbox{\scriptsize \it P}} + \mbox{\it ADP} \rightleftarrows \mbox{\it E} + \mbox{\it S}^{\mbox{\scriptsize\it P}} + \mbox{\it ADP},
$$
with conservation condition $\mbox{\it S}_{\mbox{\scriptsize\it tot}} = \mbox{\it S} + \mbox{\it S}^{\mbox{\scriptsize\it P}}$.\\
The same reaction is abstracted in our model by the corresponding interaction
$$
\mbox{\it E} \rightarrow \mbox{\it S}.
$$
Each occurrence of the interaction $E \rightarrow S$ will increase the activity level of $S$ by one discrete step.
The rate $R$ of occurrence of an interaction is defined by the user, who chooses a simplified interaction scenario
and a scale factor $k$.
The three available scenarios are based on abstract kinetic formulae:
\begin{enumerate}
  \item $R = k \times [E]$: the rate of occurrence depends only on the activity level of the upstream node
  \item $R = k \times [E] \times [\mbox{\it S}_{\mbox{\scriptsize\it tot}} - S^{\mbox{\scriptsize\it P}}]$: the rate depends on the activity levels of both participants
(note the dependence on the concentration of inactive substrate)
  \item $R = k \times [E_1] \times [E_2]$: the rate depends on the activity levels of two user-selected reactants.
This formula can be used to represent the so-called
\emph{AND gate} kinetics, i.e. the case when the activity of a downstream node depends on the
simultaneous presence of two upstream nodes.
\end{enumerate}
The value of the constant $k$ can be either given numerically, or chosen among a pre-defined set of
qualitative values, choosing from ``very slow'', ``slow'', ``normal'', ``fast'' and ``very fast''.

We will show in Section~\ref{sec:results} that the abstraction proposed here preserves ample
descriptiveness to capture experimental data in meaningful models.


\subsection{Modelling interactions with Timed Automata}\label{subsec:timed-automata}
\def\ta{TA}
\def\tas{TA}

The formal model we propose here is based on Timed Automata~(\tas, \citealt{timed-automata-alur-dill}), which is a
formalism developed for modelling dynamic concurrent systems in computer science where timing plays a central role.
Being compositional in nature, \tas\ have been shown to be a suitable solution for
modelling biological processes~\citep{ta-siebert,bartocci-oscillators,oded-ode-ta-discretization}.
Our model is based on the abstraction presented in Sect.~\ref{subsec:abstractions}
and uses the clock mechanism of \tas\ to represent interaction rates:
a faster interaction will take less time than a slower interaction to perform one activation/inhibition step.

Figure~\ref{fig:abstraction-mek-erk} contains a simplified \tas\ model of the activation of extracellular
regulated kinase (ERK) by MAPK ERK kinase (MEK) that illustrates the basic properties of \tas.



\def\mekTA{\includegraphics[scale=.098]{images/abstraction_ta_erk3}}
\newlength\mekTAheight
\setlength\mekTAheight{\heightof{\mekTA}}
\begin{figure}[!hb]
%\begin{minipage}{\textwidth}
\begin{center}
\subfloat[\label{subfig:mek-erk}]{\begin{minipage}[c][\mekTAheight]{0.13\textwidth}\begin{center}\includegraphics[scale=.098]{images/abstraction_ta_mek-erk3}\end{center}\end{minipage}}
\qquad
\subfloat[\label{subfig:mek}]{\begin{minipage}[c][\mekTAheight]{0.13\textwidth}\begin{center}\includegraphics[scale=.098]{images/abstraction_ta_mek}\end{center}\end{minipage}}
\qquad
\subfloat[\label{subfig:erk}]{\begin{minipage}[c][\mekTAheight]{0.14\textwidth}\begin{center}\mekTA\end{center}\end{minipage}}
\end{center}
\caption{Parallel between an abstract activation interaction and its \tas\ model.
{\bf \protect\subref{subfig:mek-erk}}~Classical depiction of a well-studied intracellular signal transduction interaction: protein
MAPK-ERK kinase (MEK, active) activates downstream protein extracellular-regulated kinase (ERK, initially inactive).
In this example, MEK activity is not regulated and ERK has three activity levels,
completely inactive, halfway active and completely active.
{\bf \protect\subref{subfig:mek}}~A \ta\ model of active MEK, consisting of one location (circle) and one
transition (arrow). ${\sf t} < 20$ is termed an invariant on the location, allowing residence in this location as long as local
clock time {\sf t} is smaller than $20$ units. ${\sf t} > 18$ is termed a guard on the transition, allowing the
transition to take place when local clock {\sf t} is greater than $18$ units. Together, the invariant and guard in this
example ensure that the transition must take place within the (continuous) time interval $18 < {\sf t} < 20$. When the
transition takes place, the action {\sf activate\_Erk!} is performed (thus allowing the ERK automaton to reach the {\sf
50\%\_active\_ERK} location) and the local clock coupled to this automaton is reset, ${\sf t} := 0$.
This allows the reaction to occur another time, fully activating ERK.
{\bf \protect\subref{subfig:erk}}~A \ta\ model of ERK, consisting of three locations, {\sf inactive\_ERK}
(the starting location), {\sf 50\%\_active\_ERK} and {\sf 100\%\_active\_ERK},
and two transitions between the locations. A transition will take place when it is possible to synchronize with
the corresponding action {\sf activate\_ERK!} in the MEK automaton.
Each synchronization on channel {\sf activate\_ERK} represents the occurrence of the activating
interaction between MEK and ERK, and allows ERK to eventually become completely active.}\label{fig:abstraction-mek-erk}
%\end{minipage}
\end{figure}





\subsection{ANIMO}
The modelling approach described in Section~\ref{subsec:abstractions} is implemented in the
software tool ANIMO (Analysis of Networks with Interactive MOdelling, \citealt{animo-bibe}),
a plug-in to the network visualization tool Cytoscape~\citep{cytoscape}. The visual interface of ANIMO
is designed to allow the user to intuitively construct activity-based models, making the expansion
and rewiring of a network a fast and user-friendly process (see Suppl. Video 1). An ANIMO model can
be analysed through simulation, with the results automatically plotted as a graph.
The dynamic behaviour of a model can then be interactively explored by
acting on a slider to highlight time points in a simulation. The selected simulation
point defines the appearance of the network: each node will be coloured depending on its activity level
at the selected simulation instant. Moreover, experimental data can be matched against
the predictions of a model, superposing time-based activity series to a graph produced from the model.


Each network built with ANIMO is automatically translated to
a \tas\ model, which is then simulated with the model checking tool UPPAAL~\citep{uppaal},
translating back the results in the proper user-friendly format
(all parts of Figure~\ref{fig:small-model}, as well as similar figures
in the rest of the paper, were taken from ANIMO's user interface).
No training
or prior knowledge on the use of \tas\ or UPPAAL is needed in order to benefit from ANIMO.
Nevertheless, the translation process can be followed in a transparent manner if
desired by the user.

The technical foundations of ANIMO have been described by us elsewhere~\citep{animo-bibe}. 
An abstract overview on the \tas\ model can be found in Supplementary Section~\ref{suppl-sec:animo-ta}.


\subsection{Using ANIMO to build a model based on data}\label{subsec:case-study}
To illustrate the process of model building in ANIMO, we consider now an example based on a literature compendium of
signal transduction events in HT-29 human colon carcinoma cells~\citep{pathway-compendium}. This data set comprises triplicate
measurements of 11 different protein activities or post-translational modification states at 13 time points after
treatment with different combinations of tumour necrosis factor-$\alpha$ (TNF$\alpha$), epidermal growth factor (EGF) and insulin.
The data set contains relative protein levels and activities, and no absolute quantities, which is the typical situation in biochemistry.

As a first step, we normalized measurements for each protein to the
maximum value in the complete experiment, resulting in a nondimensionalized data set that is suitable for use with ANIMO.


In Figure~\ref{fig:small-model}, we show the stepwise construction of a model of a small part of the network that is
able to account for measured variations in activity of inhibitor of nuclear factor kappa-B kinase (IKK), c-Jun N-terminal kinase-1 (JNK1),
mitogen-activated protein kinase-activated protein kinase 2 (MK2), Caspase 8 (Casp8) and Caspase 3 (Casp3) upon stimulation with 100
ng/ml TNF$\alpha$. In this example we aimed for inclusion of a minimum number of nodes in the network, while preserving biological relationships.
Multi-step cascades were aggregated into a single step when possible. Parameters for all reactions were set manually, resulting in a close
match between the model and the patterns observed in the dataset.

A more comprehensive version of this model is presented in Section~\ref{subsec:case-study-larger}.

\def\modelGraphScale{0.2}%0.148}%0.215}
\def\legendGraphScale{0.23}%0.16}
\def\halfGraphScale{0.09}%0.067}%0.1075}
\begin{figure}[!bhtp]
\centering
\begin{tabular}{ll}
\subfloat{\includegraphics[scale=\legendGraphScale]{images/small-model-1g_legenda}}\addtocounter{subfigure}{-1}\subfloat[\label{fig:small-model-first}]{\includegraphics[scale=\modelGraphScale]{images/00-paper-model1f}}
& \subfloat[\label{fig:small-model-first-graph}]{\includegraphics[scale=\halfGraphScale]{images/00-paper-graph1m_riga}} \\[5ex]
\subfloat[\label{fig:small-model-third}]{\qquad\includegraphics[scale=\modelGraphScale]{images/00-paper-model3f}}
& \subfloat[\label{fig:small-model-third-graph}]{\includegraphics[scale=\halfGraphScale]{images/00-paper-graph3n_riga}} \\[5ex]
\subfloat[\label{fig:small-model-fourth}]{\includegraphics[scale=\modelGraphScale]{images/00-paper-model4g}}
& \subfloat[\label{fig:small-model-fourth-graph}]{\includegraphics[scale=\halfGraphScale]{images/00-paper-graph4o_riga}}
\end{tabular}
  \caption{
Incremental construction of an ANIMO model of signal transduction
events in human colon carcinoma cells upon stimulation with 100 ng/ml TNF$\alpha$.
Each construction step (top to bottom) is simulated in ANIMO, giving intermediate feedback
useful for the piecewise refinement of the model.
The graphs on the right show the dynamic behaviour of the corresponding models on the left, comparing it to the measured
activity values by~\cite{pathway-compendium} (error bars represent the standard deviation).
On the vertical axis, ``100'' represents the maximum protein activity in the complete experiment.
A red vertical line in each graph highlights a selected time point in the time course:
nodes in the corresponding model are coloured according to their activity at that time point.
{\bf (\protect\subref*{fig:small-model-first}, \protect\subref*{fig:small-model-first-graph})}~Basic model showing direct activation of JNK1 and MK2 by TNF$\alpha$.
No peak dynamics are observed because no inactivating processes are present.
{\bf (\protect\subref*{fig:small-model-third}, \protect\subref*{fig:small-model-third-graph})}~The model after addition of inactivating phosphatases and a
negative feedback loop that down-regulates TNFR (TNF receptor). Note that adding TNFR internalization or phosphatases alone would not be enough to reproduce activity peaks.
{\bf (\protect\subref*{fig:small-model-fourth}, \protect\subref*{fig:small-model-fourth-graph})}~The model after addition of IKK, IL1-secretion (abstracting
the autocrine IL-1 signalling described by~\citealp{pathway-autocrine}), Casp8 and Casp3, showing the late response to TNF$\alpha$ signalling.
As the data set did not contain values for cleaved caspase-3, but only for its non-cleaved precursor pro-caspase-3,
we computed the {\sf Casp3\_{}data} series as $100\% - [\mbox{\sf pro-Casp3}]$.}\label{fig:small-model}
\end{figure}
