In living cells, most processes are regulated by networks of interacting molecules.
Aberrations in these networks underlie a wide range of pathologies,
which is a major driving force to understand the functioning of these networks.
An understanding of the associated network dynamics is crucial for understanding
their biological functioning and a thorough insight is only possibly by studying
the ensemble of molecules involved, instead of individual molecules.
The presence of both crosstalk between pathways and feedback loops further contributes to the complexity
of these networks.
The human brain is ill-suited to grasp the non-linear dynamics of these complex networks and
the entailed emergent properties, which is why computational support has a growing role in molecular biology.

The systems biology approach to understanding biological systems starts off from a
scientific question and then follows an empirical cycle \--\ or rather a positive spiral \--\ of
knowledge/theory $\rightarrow$ model $\rightarrow$ hypotheses $\rightarrow$ experiments $\rightarrow$
observations $\rightarrow$ update and/or refinement of knowledge/theory,
until an answer to the original question is found (Figure~\ref{fig:empirical-spiral}).
The model plays a pivotal role in this cycle:
\begin{enumerate}
  \item to organize data and store knowledge,
  \item to structure reasoning  and discussion
  \item to perform in silico experiments and derive hypotheses.
\end{enumerate}
The model is always a simplified representation of biological reality and is never the aim in itself.
Rather it is a powerful means in the process of gaining an understanding of the biological system
under study. Given its role in the empirical cycle, the process of modelling is especially effective
when applied by the experts with respect to a certain biological system. Biologists usually have a good
sense of cause-and-effect relationships between molecules in the biological system they are studying.
As they also have extensive knowledge on the network topology and the dynamics of the system, they
would be the primary candidates to construct models of their research topic.

\begin{figure}[!htb]
  \centering
  \includegraphics[width=0.35\textwidth]{images/empirical_spiral4}
 \caption{The empirical spiral.}\label{fig:empirical-spiral}
\end{figure}


As models are a formalization of knowledge/theories, an underlying formalism is needed to express
this knowledge. Different formal methods have been successfully applied to construct representations
of biological systems. Among these methods are ordinary differential equations (ODEs, reviewed by~\citealp[]{hidde-review}),
process calculi~\citep{blenx,bio-pepa}, interacting state machines~\citep{interacting-sm1,interacting-sm2},
Petri nets~\citep{petri-nets,petri-nets2}, Boolean logic~\citep{boolean-networks-flower,boolean-networks2}.
%and Timed Automata~\citep{ta-siebert,bartocci-oscillators,oded-ode-ta-discretization}.

\tas\ have been previously suggested for application in a biological context.
\cite{ta-siebert} present a way to extend a classical modeling paradigm
that was introduced previously~\citep{thomas-formalism}, allowing to add time informations to gene network models.
\cite{bartocci-oscillators} describe a model of oscillators that accounts for time
dynamics and test synchronization properties of biological oscillators.
A discretization of ODEs to \tas\ is proposed by~\citet{oded-ode-ta-discretization}, applying
the formal translation to an example gene regulatory network. Two different approaches to transforming
a Petri net~\citep{petri-nets} model into \tas\ are presented by~\citet{ta-giapponesi},
where also the important issue of state space explosion is addressed.
Finally, \cite{ta-radiazioni} propose an \emph{ad hoc} \tas\ model of a radiation treatment
system, which is then validated through UPPAAL.\\
Each of these approaches has been successfully validated and demonstrates the power of \tas,
both on the theoretical level and in real biological applications. However, none of the listed approaches
has led to a tool implementation of the proposed method, the application of which was often limited to simple
or specific examples. The challenges of a broader applicability of \tas\ were not previously considered,
nor was the usability of the approach evaluated from the point of view of a biologist.


Most of the other cited formal methods have been implemented into software tools to aid the process of modelling.
However, mastery of these tools requires
training and experience in mathematical modelling. In this respect, a lack of tradition in quantitative
reasoning and formal methods within the biological community at large is still a stumbling block for
widespread application of modelling of biological systems. Here, we present an intuitive method for the
construction of formal in silico models of the dynamics of molecular networks, supported by a novel,
user friendly modelling tool, ANIMO (Analysis of Networks with Interactive MOdeling,~\citealp[]{animo-bibe}).

In the Methods section, we will explain how choosing a suitable abstraction level can make the construction
of models more intuitive. We will then show how ANIMO is designed to support the modelling process following
this approach. Construction of a small model based on experimental data will exemplify the method that we
propose. In the Results section, we first show an ANIMO model of the genes and proteins that constitute
the circadian clock network in Drosophila Melanogaster. The remainder of that section is dedicated to
illustrate how a single modelling iteration in the empirical cycle is used to compile prior knowledge and
experimental data into a model and derive meaningful biological hypotheses from this model. These hypotheses
find ample support in literature.
