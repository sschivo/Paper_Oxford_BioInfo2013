The ultimate aim of research projects is to solve a problem or get
the answer to a scientific question. In biology, an in-depth understanding of
the relevant biological system is an important step towards this goal. Successive 
repetitions of the empirical cycle result in a stepwise increase in understanding,
until the goal is reached. For complex biological systems, computational modelling is 
indispensable in this process. The mere act of creating 
a computational model based on prior knowledge, experimental data and hypotheses 
assist in gaining more insight in the system. 

%Furthermore, \emph{in silico} experiments 
%with the model can be used to make predictions that cannot be made by the human brain. 

We developed a new modelling approach by proposing a series of abstractions from the detailed 
molecular mechanisms of biological systems. These abstraction reduce the need for kinetic 
parameters, while preserving enough expresivity for a useful description of the dynamic 
behavior of biological networks. A novel modeling tool, ANIMO, allows 
effective use of this approach and enables an intuitive construction of formal models.

ANIMO is designed to make modelling more accessible to experts in biology,
without the need to acquire additional training in a modeling formalism. Making use of the visual
interface provided by Cytoscape, network representations are following biological conventions. The
underlying formal model can be directly accessed by mouse-clicking on nodes and edges.



ANIMO is not the first modeling tool to provide an interface to a
modelling formalism.: that is already successfully done in many other tools.
In Supplementary Section~\ref{suppl:comparison-table} we show a
comparison between ANIMO and other tools for modelling biological networks.

In Section~\ref{sec:animo-drosophila}, we described the construction of an ANIMO
model with a behaviour similar to an ODE model that had been
published previously~\citep{drosophila-ode-model}. The biggest
difference between the construction of these models is that the latter
is constructed by writing a system of mathematical equations, together
with an algorithm for simulation. Drawing a number of network
nodes for the molecules involved and linking these nodes with directed
cause-and-effect relationships is a more intuitive way of model
construction. A single parameter then defines the strength of each
interaction. Further contributing to an interactive modelling process
is the compositionality of the model. Each node in the network
can be disabled at any time by the user, or extra nodes can be added,
without having to change any of the existing interactions.

In Section~\ref{subsec:case-study-larger}, we showed the construction of an executable model
of signalling events downstream of
TNF$\alpha$ and EGF in human colon carcinoma cells. The same data set has been used for
previous modelling studies, based on partial least-squares regression and fuzzy logic~\citep{pathway-leastsquare,pathway-fuzzy}.
The partial least-squares regression model describes an abstract data-driven model that uses statistical correlations
to relate signal transduction events to various cellular decisions. This type of modelling is
very useful in uncovering new and unexpected relations. It is also successful in making
predictions, but gives little direct insight in dynamic behaviour. Fuzzy
logic analysis led to a model that gave a better fit to the dynamic network behaviour than
discrete logic (Boolean) models. Inspection of the inputs to the logical gates that were used
to model protein behaviour led to the prediction of novel interactions between proteins,
showing the usefulness of this approach. For most of the proteins, such as JNK1, time was
used as an input parameter. For example, the logical gates ``if TNF$\alpha$ is high
\emph{AND} time is low, then JNK1 is high'' and ``if TNF$\alpha$ is high \emph{AND} time is
high, then JNK1 is low'' were used to
describe the dynamic behaviour of JNK1. Although this leads to a useful and representative
description of the dynamic behaviour of JNK1, it gives no insight in the dynamics
of activation and inactivation. Furthermore, in many cases peaks in protein activity were
not reproduced by the fuzzy logic model.

In contrast with the two approaches described above, ANIMO is aimed at the construction of
more mechanistic models, mimicking biochemical interactions \emph{in silico}. This way of modelling
gives a different type of insight. In the process of model construction, we extended a
prior-knowledge network with time-dependent extracellular crosstalk that has been reported
previously~\citep{pathway-autocrine}. We then introduced two additional layers of
crosstalk, at the signal transduction and transcriptional level, to
improve the fit of the model to the data. These modifications can be interpreted as novel testable hypotheses,
and are supported in literature.
Together, our model sheds more light on the intricate
entanglement between the TNF$\alpha$ and EGF pathways at multiple cellular levels and will
provide a starting point for further investigation.

ANIMO is leading to a new paradigm for interactive
representation of biological networks. Networks in digital textbooks and articles can be
displayed as animations amenable to modifications by readers. Repositories of formal
descriptions of signalling modules could be used to put together executable signalling
networks. A more user-friendly way of interacting with dynamic network models will lead to a
more thorough understanding of biological networks and will accelerate hypothesis-driven
research.
