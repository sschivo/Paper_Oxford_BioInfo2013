The ultimate aim of research projects is to solve a problem or get
the answer to a scientific question. In biology, an in-depth understanding of
the relevant biological system is an important step towards this goal. Successive 
repetitions of the empirical cycle result in a stepwise increase in understanding,
until the goal is reached. For complex biological systems, computational modelling is 
indispensable in this process. The mere act of creating 
a computational model based on prior knowledge, experimental data and hypotheses 
assists in gaining more insight in the system. 

%Furthermore, \emph{in silico} experiments 
%with the model can be used to make predictions that cannot be made by the human brain. 

We developed a new modelling approach by proposing a series of abstractions from the detailed 
molecular mechanisms of biological systems. These abstractions reduce the need for kinetic 
parameters, while preserving enough expressivity for a useful description of the dynamic 
behaviour of biological networks. A novel modelling tool, ANIMO, allows 
effective use of this approach and enables an intuitive construction of formal models.

ANIMO is not the first modelling tool to provide an interface to a
modelling formalism. Such interfaces exist in many other tools (see Suppl. Tab.~\ref{tab:tool-comparison}). With its
focus on user-friendliness and intuitive modelling, ANIMO's main contribution lies 
in making computational modelling more accessible to experts in biology.
Making use of the visual
interface provided by Cytoscape, network representations subscribe to biological conventions. 
Model parameters are kept to a minimum and can be directly accessed by mouse-clicking on 
nodes and edges. Because of the automatic translation of the network topology and 
user-defined parameters into an underlying formal model, training in the use of formal methods 
is not needed. In Supplementary Section~\ref{suppl:comparison-table}, a more in-depth
comparison between ANIMO and other modeling tools is given. For this comparison we selected a tool
for each of the most commonly used formalisms, and used criteria with a strong focus on 
user-friendliness.

In Section~\ref{sec:animo-drosophila}, we described the construction of an ANIMO
model of the ciracadian clock in \emph{Drosophila Melanogaster}. This model
captured the dynamics of the regualtory network and led to similar 
conclusions as an ODE model that had been
published previously~\citep{drosophila-ode-model}. This finding supports the use of
the series of modelling abstractions that we proposed. The biggest
difference between the construction of these models is that the model by~\citet{drosophila-ode-model}
is constructed by writing a system of mathematical equations, together
with an algorithm for simulation. In ANIMO, instead, a number of network
nodes is drawn for the molecules involved. 
These nodes are then linked by directed
interactions that represent cause-and-effect relationships, with a single parameter 
that defines the strength of each
interaction. This is a more intuitive approach to construct a model.
Further contributing to an interactive modelling process
is the compositionality of the model. Each node in the network
can be disabled at any time by the user, or extra nodes can be added,
without having to change any of the existing interactions.

In Section~\ref{subsec:case-study-larger}, we showed the construction of an executable model
of signalling events downstream of
TNF$\alpha$ and EGF in human colon carcinoma cells. This data set has been used for
previous modelling studies, based on partial least-squares regression and fuzzy 
logic~\citep{pathway-leastsquare,pathway-fuzzy}.
The partial least-squares regression model describes an abstract data-driven model 
that uses statistical correlations
to relate signal transduction events to various cellular decisions. This type of modelling is
very useful in uncovering new and unexpected relations. It is also successful in making
predictions, but gives little direct in the dynamic behaviour of the network. Fuzzy
logic analysis led to a model that gave a better fit to the dynamic network behaviour than
discrete logic (Boolean) models. Inspection of the inputs to the logical gates that were used
to model protein behaviour led to the prediction of novel interactions between proteins,
showing the usefulness of this approach. For most of the proteins, such as JNK1, time was
used as an input parameter. For example, the logical gates ``if TNF$\alpha$ is high
\emph{AND} time is low, then JNK1 is high'' and ``if TNF$\alpha$ is high \emph{AND} time is
high, then JNK1 is low'' were used to
describe the dynamic behaviour of JNK1. Although this leads to a representative
description of the dynamic behaviour of JNK1, peaks in protein activity at early time points were
not reproduced by the fuzzy logic model. Moreover, it gives no insight in the molecular interactions 
that are involved in activation or inhibition.

In this study, we used the same data set and performed a single round of the empirical cycle. This cycle starts off with the experiments carried out by~\citet{pathway-compendium}. We used the resulting experimental data, together with knowledge from curated databases~\citep{kegg,phosphosite} to construct an executable model of the biological system.
In contrast with the two approaches described above, ANIMO is aimed at the construction of
more mechanistic models, mimicking biochemical interactions \emph{in silico}. This way of modelling
gives a different type of insight. In the process of model construction, we extended a
prior-knowledge network with time-dependent extracellular crosstalk that has been reported
previously~\citep{pathway-autocrine}. To come up with possible explanations for a disagreement
between experimental data, two additional layers of
crosstalk were introduced, at the signal transduction and transcriptional level. These modifications 
improved the fit of the model to the data and can be interpreted as novel testable hypotheses.
Finally, we proposed new experiments that could be carried out to test these hypotheses, closing the empirical cycle. 
Together, our model sheds more light on the intricate
entanglement between the TNF$\alpha$ and EGF pathways at multiple cellular levels.
But above all,  the model provides an excellent starting point for further investigation.
Every new round in the empirical cycle will lift the understanding of the system to a higher level, leading to an incremental build-up of knowledge and an upward empirical spiral. Being intuitively accessible, ANIMO models facilitate sharing knowledge within and between groups and encourage collaborations.
