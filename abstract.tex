\section{Motivation:}
Computational modelling is a vital part of the empirical cycle,
formalizing existing knowledge and experimental data in a systematic way.
Existing methods are often parameter intensive, or supporting tools are
not accessible enough to the biological community. These factors
hamper widespread application of modelling in biological research.
\section{Results:}
We present an approach to modelling based on a series of
abstractions that limit the number of parameters in the model.
To support this approach, we developed a user-friendly software tool,
ANIMO (Analysis of Networks with Interactive MOdelling).
An ANIMO model of signal transduction events downstream of TNF$\alpha$
and EGF in HT-29 human colon carcinoma cells is described as a case study.
This model gives a formal description of crosstalk between the pathways at
different cellular levels and led to the formulation of novel testable hypotheses.
\section{Availability and implementation:}
ANIMO is implemented as a plug-in to the network visualization tool Cytoscape
and can be downloaded from \url{http://fmt.cs.utwente.nl/tools/animo}.
Additional information and a step-by-step installation guide can be found
in Supplementary Section~\ref{sec:animo-installation}. A user manual is 
provided in Supplementary Section~\ref{sec:animo-manual}.
\section{Contact:} \{j.c.vandepol, j.n.post\}@utwente.nl
